\PassOptionsToPackage{unicode=true}{hyperref} % options for packages loaded elsewhere
\PassOptionsToPackage{hyphens}{url}
%
\documentclass[]{article}
\usepackage{lmodern}
\usepackage{amssymb,amsmath}
\usepackage{ifxetex,ifluatex}
\usepackage{fixltx2e} % provides \textsubscript
\ifnum 0\ifxetex 1\fi\ifluatex 1\fi=0 % if pdftex
  \usepackage[T1]{fontenc}
  \usepackage[utf8]{inputenc}
  \usepackage{textcomp} % provides euro and other symbols
\else % if luatex or xelatex
  \usepackage{unicode-math}
  \defaultfontfeatures{Ligatures=TeX,Scale=MatchLowercase}
\fi
% use upquote if available, for straight quotes in verbatim environments
\IfFileExists{upquote.sty}{\usepackage{upquote}}{}
% use microtype if available
\IfFileExists{microtype.sty}{%
\usepackage[]{microtype}
\UseMicrotypeSet[protrusion]{basicmath} % disable protrusion for tt fonts
}{}
\IfFileExists{parskip.sty}{%
\usepackage{parskip}
}{% else
\setlength{\parindent}{0pt}
\setlength{\parskip}{6pt plus 2pt minus 1pt}
}
\usepackage{hyperref}
\hypersetup{
            pdftitle={Advanced Remote Sensing Lab 2: Introduction to Google Earth Engine in R},
            pdfauthor={Victor Gutierrez (victorhugo@temple.edu)},
            pdfborder={0 0 0},
            breaklinks=true}
\urlstyle{same}  % don't use monospace font for urls
\usepackage[margin=1in]{geometry}
\usepackage{color}
\usepackage{fancyvrb}
\newcommand{\VerbBar}{|}
\newcommand{\VERB}{\Verb[commandchars=\\\{\}]}
\DefineVerbatimEnvironment{Highlighting}{Verbatim}{commandchars=\\\{\}}
% Add ',fontsize=\small' for more characters per line
\usepackage{framed}
\definecolor{shadecolor}{RGB}{248,248,248}
\newenvironment{Shaded}{\begin{snugshade}}{\end{snugshade}}
\newcommand{\AlertTok}[1]{\textcolor[rgb]{0.94,0.16,0.16}{#1}}
\newcommand{\AnnotationTok}[1]{\textcolor[rgb]{0.56,0.35,0.01}{\textbf{\textit{#1}}}}
\newcommand{\AttributeTok}[1]{\textcolor[rgb]{0.77,0.63,0.00}{#1}}
\newcommand{\BaseNTok}[1]{\textcolor[rgb]{0.00,0.00,0.81}{#1}}
\newcommand{\BuiltInTok}[1]{#1}
\newcommand{\CharTok}[1]{\textcolor[rgb]{0.31,0.60,0.02}{#1}}
\newcommand{\CommentTok}[1]{\textcolor[rgb]{0.56,0.35,0.01}{\textit{#1}}}
\newcommand{\CommentVarTok}[1]{\textcolor[rgb]{0.56,0.35,0.01}{\textbf{\textit{#1}}}}
\newcommand{\ConstantTok}[1]{\textcolor[rgb]{0.00,0.00,0.00}{#1}}
\newcommand{\ControlFlowTok}[1]{\textcolor[rgb]{0.13,0.29,0.53}{\textbf{#1}}}
\newcommand{\DataTypeTok}[1]{\textcolor[rgb]{0.13,0.29,0.53}{#1}}
\newcommand{\DecValTok}[1]{\textcolor[rgb]{0.00,0.00,0.81}{#1}}
\newcommand{\DocumentationTok}[1]{\textcolor[rgb]{0.56,0.35,0.01}{\textbf{\textit{#1}}}}
\newcommand{\ErrorTok}[1]{\textcolor[rgb]{0.64,0.00,0.00}{\textbf{#1}}}
\newcommand{\ExtensionTok}[1]{#1}
\newcommand{\FloatTok}[1]{\textcolor[rgb]{0.00,0.00,0.81}{#1}}
\newcommand{\FunctionTok}[1]{\textcolor[rgb]{0.00,0.00,0.00}{#1}}
\newcommand{\ImportTok}[1]{#1}
\newcommand{\InformationTok}[1]{\textcolor[rgb]{0.56,0.35,0.01}{\textbf{\textit{#1}}}}
\newcommand{\KeywordTok}[1]{\textcolor[rgb]{0.13,0.29,0.53}{\textbf{#1}}}
\newcommand{\NormalTok}[1]{#1}
\newcommand{\OperatorTok}[1]{\textcolor[rgb]{0.81,0.36,0.00}{\textbf{#1}}}
\newcommand{\OtherTok}[1]{\textcolor[rgb]{0.56,0.35,0.01}{#1}}
\newcommand{\PreprocessorTok}[1]{\textcolor[rgb]{0.56,0.35,0.01}{\textit{#1}}}
\newcommand{\RegionMarkerTok}[1]{#1}
\newcommand{\SpecialCharTok}[1]{\textcolor[rgb]{0.00,0.00,0.00}{#1}}
\newcommand{\SpecialStringTok}[1]{\textcolor[rgb]{0.31,0.60,0.02}{#1}}
\newcommand{\StringTok}[1]{\textcolor[rgb]{0.31,0.60,0.02}{#1}}
\newcommand{\VariableTok}[1]{\textcolor[rgb]{0.00,0.00,0.00}{#1}}
\newcommand{\VerbatimStringTok}[1]{\textcolor[rgb]{0.31,0.60,0.02}{#1}}
\newcommand{\WarningTok}[1]{\textcolor[rgb]{0.56,0.35,0.01}{\textbf{\textit{#1}}}}
\usepackage{graphicx,grffile}
\makeatletter
\def\maxwidth{\ifdim\Gin@nat@width>\linewidth\linewidth\else\Gin@nat@width\fi}
\def\maxheight{\ifdim\Gin@nat@height>\textheight\textheight\else\Gin@nat@height\fi}
\makeatother
% Scale images if necessary, so that they will not overflow the page
% margins by default, and it is still possible to overwrite the defaults
% using explicit options in \includegraphics[width, height, ...]{}
\setkeys{Gin}{width=\maxwidth,height=\maxheight,keepaspectratio}
\setlength{\emergencystretch}{3em}  % prevent overfull lines
\providecommand{\tightlist}{%
  \setlength{\itemsep}{0pt}\setlength{\parskip}{0pt}}
\setcounter{secnumdepth}{0}
% Redefines (sub)paragraphs to behave more like sections
\ifx\paragraph\undefined\else
\let\oldparagraph\paragraph
\renewcommand{\paragraph}[1]{\oldparagraph{#1}\mbox{}}
\fi
\ifx\subparagraph\undefined\else
\let\oldsubparagraph\subparagraph
\renewcommand{\subparagraph}[1]{\oldsubparagraph{#1}\mbox{}}
\fi

% set default figure placement to htbp
\makeatletter
\def\fps@figure{htbp}
\makeatother


\title{Advanced Remote Sensing Lab 2: Introduction to Google Earth Engine in R}
\author{Victor Gutierrez
(\href{mailto:victorhugo@temple.edu}{\nolinkurl{victorhugo@temple.edu}})}
\date{}

\begin{document}
\maketitle

\hypertarget{lab-due}{%
\subsection{Lab due}\label{lab-due}}

January 26 2021

\hypertarget{goals}{%
\subsection{Goals}\label{goals}}

\begin{enumerate}
\def\labelenumi{\arabic{enumi}.}
\tightlist
\item
  To become familiar with the layout and functionality of the GEE
  platform.
\item
  To become familiar with basic JavaScript coding in general and applied
  to GEE
\item
  To introduce available data, basic object types and manipulation in
  GEE
\end{enumerate}

\hypertarget{total-score}{%
\subsection{Total score}\label{total-score}}

The lab counts for up to 3 points towards the final grade of the course.

Have fun!

\hypertarget{lab-instructions}{%
\subsection{Lab instructions}\label{lab-instructions}}

\begin{enumerate}
\def\labelenumi{\arabic{enumi}.}
\item
  Launch Google Earth engine in your browser (preferably chrome):
  \url{https://code.earthengine.google.com/}
\item
  Go to the help menu (? symbol in the top-left of the interface) and
  perform a ``Feature Tour''
\end{enumerate}

\includegraphics[width=21.11in]{/Users/tug61163/Documents/Courses/AdvancedRS/Spring2021/Session2_IntroToGEE/GEEHelp}

\begin{enumerate}
\def\labelenumi{\arabic{enumi}.}
\setcounter{enumi}{2}
\tightlist
\item
  Click on the ``Data Sets'' tab at the top of the search panel. Then on
  the drop down menu select Landsat/Landsat Collection 1 Level-1/Landsat
  8 OLI/TIRS C1 Level-1. Browse the datasets available
\end{enumerate}

\includegraphics[width=31.19in]{/Users/tug61163/Documents/Courses/AdvancedRS/Spring2021/Session2_IntroToGEE/GEECatalog}

Copy and paste the code below

\begin{Shaded}
\begin{Highlighting}[]
\CommentTok{// Get the image.}
\KeywordTok{var}\NormalTok{ lc8 }\OperatorTok{=} \VariableTok{ee}\NormalTok{.}\AttributeTok{Image}\NormalTok{(}\StringTok{'LANDSAT/LC8_L1T_TOA/LC80140322016081LGN00'}\NormalTok{)}

\CommentTok{// Add the image to the map.}
\VariableTok{Map}\NormalTok{.}\AttributeTok{addLayer}\NormalTok{(lc8)}\OperatorTok{;}

\CommentTok{// Center the map display on the image.}
\VariableTok{Map}\NormalTok{.}\AttributeTok{centerObject}\NormalTok{(lc8}\OperatorTok{,} \DecValTok{8}\NormalTok{)}\OperatorTok{;}
\end{Highlighting}
\end{Shaded}

You can see that the image looks very dark. Can you explain why?

\begin{enumerate}
\def\labelenumi{\arabic{enumi}.}
\setcounter{enumi}{3}
\tightlist
\item
  Check the parameterization of the addLayer() function by searching on
  the Docs tab. Notice the name of the second parameter (visparams).
  Check the visualization parameters here:
  \url{https://developers.google.com/earth-engine/guides/image_visualization}
\end{enumerate}

Use either of the options below to visualize the image. Modify visparams
to improve the visualization as shown below. You can read about gamma
correction here:
\url{https://www.cambridgeincolour.com/tutorials/gamma-correction.htm}.
Add also a name to the layer.

\begin{Shaded}
\begin{Highlighting}[]
\CommentTok{//OPTION 1: show layers individually}
\VariableTok{Map}\NormalTok{.}\AttributeTok{addLayer}\NormalTok{(lc8}\OperatorTok{,}
\OperatorTok{\{}\DataTypeTok{bands}\OperatorTok{:} \StringTok{'B6, B5, B4'}\OperatorTok{,} \DataTypeTok{min}\OperatorTok{:} \FloatTok{0.05}\OperatorTok{,} \DataTypeTok{max}\OperatorTok{:} \FloatTok{0.8}\OperatorTok{,} \DataTypeTok{gamma}\OperatorTok{:} \FloatTok{1.6}\OperatorTok{\},} \StringTok{'Landsat8TOA'}\NormalTok{)}\OperatorTok{;}

\CommentTok{//OPTION 2: show layers as a string}
\VariableTok{Map}\NormalTok{.}\AttributeTok{addLayer}\NormalTok{(lc8}\OperatorTok{,}
\OperatorTok{\{}\DataTypeTok{bands}\OperatorTok{:}\NormalTok{ [}\StringTok{'B6'}\OperatorTok{,} \StringTok{'B5'}\OperatorTok{,} \StringTok{'B4'}\NormalTok{]}\OperatorTok{,} \DataTypeTok{min}\OperatorTok{:} \FloatTok{0.05}\OperatorTok{,} \DataTypeTok{max}\OperatorTok{:} \FloatTok{0.8}\OperatorTok{,} \DataTypeTok{gamma}\OperatorTok{:} \FloatTok{1.6}\OperatorTok{\},} \StringTok{'Landsat8TOA'}\NormalTok{)}\OperatorTok{;}
\end{Highlighting}
\end{Shaded}

\hypertarget{explore-some-of-the-settings-and-the-work-environment.}{%
\section{5. Explore some of the settings and the work
environment.}\label{explore-some-of-the-settings-and-the-work-environment.}}

Change the second argument of the centerObject() function from 8 to 10
and run. What happens? What happens when you change it to 6?

Explore the map window tool by zooming in and out (red circle \#1).

Use the layers tool to turn the open layer on and off. Notice that the
name is the same as the one that you specified in the addLayer()
function (red circle \#2).

Click on the inspector, then click anywhere in the image and explore the
information that appears in the inspector. Make sure you expand each
attribute (red circle \#3).

Write the code below and explore the image properties in the console
(red circle \#4).

\begin{Shaded}
\begin{Highlighting}[]
\AttributeTok{print}\NormalTok{(lc8)}\OperatorTok{;}
\end{Highlighting}
\end{Shaded}

\includegraphics[width=46.56in]{/Users/tug61163/Documents/Courses/AdvancedRS/Spring2021/Session2_IntroToGEE/GEEWindows}
6. Calculate the NDVI for your image and display it. The "palette
argument uses a CSS style. You can check the color codes here:
\url{https://www.quackit.com/css/css_color_codes.cfm}

\begin{Shaded}
\begin{Highlighting}[]
\CommentTok{//Calculate NDVI}
\KeywordTok{var}\NormalTok{ NDVI }\OperatorTok{=} \VariableTok{lc8}\NormalTok{.}\AttributeTok{normalizedDifference}\NormalTok{([}\StringTok{'B5'}\OperatorTok{,} \StringTok{'B4'}\NormalTok{])}\OperatorTok{;}

\CommentTok{// Display image in a gradient stretch}
\VariableTok{Map}\NormalTok{.}\AttributeTok{addLayer}\NormalTok{ (NDVI}\OperatorTok{,} \OperatorTok{\{}\DataTypeTok{min}\OperatorTok{:} \FloatTok{-0.2}\OperatorTok{,} \DataTypeTok{max}\OperatorTok{:} \FloatTok{0.5}\OperatorTok{,} \DataTypeTok{palette}\OperatorTok{:}\NormalTok{ [}\StringTok{'FFFFFF'}\OperatorTok{,} \StringTok{'339900'}\NormalTok{]}\OperatorTok{\},} \StringTok{"NDVI"}\NormalTok{)}\OperatorTok{;}
\end{Highlighting}
\end{Shaded}

\begin{enumerate}
\def\labelenumi{\arabic{enumi}.}
\setcounter{enumi}{6}
\tightlist
\item
  Mask clouds. Remove the lines used to calculate the NDVI and display
  it. The code should look like this
\end{enumerate}

\begin{Shaded}
\begin{Highlighting}[]
\CommentTok{// Get the image.}
\KeywordTok{var}\NormalTok{ lc8 }\OperatorTok{=} \VariableTok{ee}\NormalTok{.}\AttributeTok{Image}\NormalTok{(}\StringTok{'LANDSAT/LC8_L1T_TOA/LC80140322016081LGN00'}\NormalTok{)}
\CommentTok{// Add the image to the map.}
\VariableTok{Map}\NormalTok{.}\AttributeTok{addLayer}\NormalTok{(lc8}\OperatorTok{,}
\OperatorTok{\{}\DataTypeTok{bands}\OperatorTok{:}\NormalTok{ [}\StringTok{'B6'}\OperatorTok{,} \StringTok{'B5'}\OperatorTok{,} \StringTok{'B4'}\NormalTok{]}\OperatorTok{,} \DataTypeTok{min}\OperatorTok{:} \FloatTok{0.05}\OperatorTok{,} \DataTypeTok{max}\OperatorTok{:} \FloatTok{0.8}\OperatorTok{,} \DataTypeTok{gamma}\OperatorTok{:} \FloatTok{1.6}\OperatorTok{\},} \StringTok{'landsat 8 TOA'}\NormalTok{)}\OperatorTok{;}

\CommentTok{// Center the map display on the image.}
\VariableTok{Map}\NormalTok{.}\AttributeTok{centerObject}\NormalTok{(lc8}\OperatorTok{,} \DecValTok{8}\NormalTok{)}\OperatorTok{;}
\end{Highlighting}
\end{Shaded}

Select the ``Docs'' tab and search for ``cloud''. Then inpsect the
function ``ee.Algorithms.Landsat.simpleCloudScore(image)''. Then write
the code below:

\begin{Shaded}
\begin{Highlighting}[]
\CommentTok{// Add the cloud likelihood band to the image.}
\KeywordTok{var}\NormalTok{ cloudscore }\OperatorTok{=} \VariableTok{ee}\NormalTok{.}\VariableTok{Algorithms}\NormalTok{.}\VariableTok{Landsat}\NormalTok{.}\AttributeTok{simpleCloudScore}\NormalTok{(lc8)}\OperatorTok{;}

\CommentTok{// Add the cloud image to the map. This will display the first three bands as R, G, B by default.}
\VariableTok{Map}\NormalTok{.}\AttributeTok{addLayer}\NormalTok{(cloudscore}\OperatorTok{,} \OperatorTok{\{\},} \StringTok{'Cloud Likelihood, all bands'}\NormalTok{)}\OperatorTok{;}
\end{Highlighting}
\end{Shaded}

Click on the ``inspector tab and click on different locations in the
image. You can see that the new cloudscore image added a band
named''clouds" to the original dataset. Let's select that band and
display it.

\begin{Shaded}
\begin{Highlighting}[]
\CommentTok{// Since you are interested in only the cloud layer, specify just this band to be displayed in the parameters of the Map.addLayer statement. }
\VariableTok{Map}\NormalTok{.}\AttributeTok{addLayer}\NormalTok{(cloudscore}\OperatorTok{,} \OperatorTok{\{}\DataTypeTok{bands}\OperatorTok{:} \StringTok{'cloud'}\OperatorTok{\},} \StringTok{'Cloud Likelihood'}\NormalTok{)}\OperatorTok{;}
\end{Highlighting}
\end{Shaded}

Select the inspector tab and then in different locations of the image to
assess how the cloud score values differ in cloudy and non-cloudy areas.
Based on your assessment, select the threshold that you think is best to
discriminate between cloudy and non-cloudy areas. Let's use that
threshold to mask clouds. In the code below, I selected 40

\begin{Shaded}
\begin{Highlighting}[]
\CommentTok{// Isolate the cloud likelihood band.}
\KeywordTok{var}\NormalTok{ cloudLikelihood }\OperatorTok{=} \VariableTok{cloudscore}\NormalTok{.}\AttributeTok{select}\NormalTok{(}\StringTok{'cloud'}\NormalTok{)}\OperatorTok{;}

\CommentTok{// Compute a mask in which pixels below the threshold are 1 (you can check the lt function in the Docs tab.}
\KeywordTok{var}\NormalTok{ cloudPixels }\OperatorTok{=} \VariableTok{cloudLikelihood}\NormalTok{.}\AttributeTok{lt}\NormalTok{(}\DecValTok{40}\NormalTok{)}\OperatorTok{;}

\CommentTok{// Add the image to the map.}
\VariableTok{Map}\NormalTok{.}\AttributeTok{addLayer}\NormalTok{(cloudPixels}\OperatorTok{,} \OperatorTok{\{\},} \StringTok{'Cloud Mask'}\NormalTok{)}\OperatorTok{;}
\end{Highlighting}
\end{Shaded}

In the ``layers'' tab uncheck all the layers except the cloud mask and
the Landsat toa layers. Select and de-select the cloud mask to compare
with the RGB and assess whether you are satisfied with the result
obtained. If not, change the threshold and run again.

Use the created mask to remove clouds from the Landsat image and check
the results

\begin{Shaded}
\begin{Highlighting}[]
\CommentTok{// Mask out the Landsat image.}
\KeywordTok{var}\NormalTok{ lc8_NoClouds }\OperatorTok{=} \VariableTok{lc8}\NormalTok{.}\AttributeTok{updateMask}\NormalTok{ (cloudPixels)}\OperatorTok{;}

\CommentTok{// Review the result.}
\VariableTok{Map}\NormalTok{.}\AttributeTok{addLayer}\NormalTok{(lc8_NoClouds}\OperatorTok{,} \OperatorTok{\{}\DataTypeTok{bands}\OperatorTok{:}\NormalTok{ [}\StringTok{'B6'}\OperatorTok{,} \StringTok{'B5'}\OperatorTok{,} \StringTok{'B4'}\NormalTok{]}\OperatorTok{,} \DataTypeTok{min}\OperatorTok{:} \FloatTok{0.1}\OperatorTok{,} \DataTypeTok{max}\OperatorTok{:} \FloatTok{0.5}\OperatorTok{\},} \StringTok{'Landsat8scene_cloudmasked'}\NormalTok{)}\OperatorTok{;}
\end{Highlighting}
\end{Shaded}

Zoom into the Philadelphia area. How did the cloud masking perform in
urban areas? Check the hazy cloud in the north-west. How well was the
hazy area masked?

\begin{enumerate}
\def\labelenumi{\arabic{enumi}.}
\setcounter{enumi}{7}
\tightlist
\item
  Perform an unsupervised classification:
\end{enumerate}

\begin{Shaded}
\begin{Highlighting}[]
\CommentTok{// select the bounding box of a Landsat-8 image to sample pixels}
\KeywordTok{var}\NormalTok{ region }\OperatorTok{=} \VariableTok{lc8_NoClouds}\NormalTok{.}\AttributeTok{geometry}\NormalTok{()}
\AttributeTok{print}\NormalTok{(region)}

\CommentTok{// Select random pixels for training the unsupervised classification}
\KeywordTok{var}\NormalTok{ training }\OperatorTok{=} \VariableTok{lc8_NoClouds}\NormalTok{.}\AttributeTok{sample}\NormalTok{(}\OperatorTok{\{}
  \DataTypeTok{region}\OperatorTok{:}\NormalTok{ region}\OperatorTok{,}
  \DataTypeTok{scale}\OperatorTok{:} \DecValTok{30}\OperatorTok{,}
  \DataTypeTok{numPixels}\OperatorTok{:} \DecValTok{5000}
\OperatorTok{\}}\NormalTok{)}\OperatorTok{;}

\CommentTok{// train cluster based on sampled pixels}
\KeywordTok{var}\NormalTok{ clusterer }\OperatorTok{=} \VariableTok{ee}\NormalTok{.}\VariableTok{Clusterer}\NormalTok{.}\AttributeTok{wekaKMeans}\NormalTok{(}\DecValTok{15}\NormalTok{).}\AttributeTok{train}\NormalTok{(training)}\OperatorTok{;}

\CommentTok{// classify the entire image based on teh trained cluster}
\KeywordTok{var}\NormalTok{ result }\OperatorTok{=} \VariableTok{lc8_NoClouds}\NormalTok{.}\AttributeTok{cluster}\NormalTok{(clusterer)}\OperatorTok{;}

\CommentTok{// Display the classified image with random colors.}
\VariableTok{Map}\NormalTok{.}\AttributeTok{addLayer}\NormalTok{(}\VariableTok{result}\NormalTok{.}\AttributeTok{randomVisualizer}\NormalTok{()}\OperatorTok{,} \OperatorTok{\{\},} \StringTok{'clusters'}\NormalTok{)}\OperatorTok{;}
\end{Highlighting}
\end{Shaded}

Explore the results and modify the number of clustes in the variable
``clusterer'' as needed until the number of classes descriminate
reasonably well the land covers in the landscape.

\hypertarget{homework}{%
\subsection{Homework}\label{homework}}

Apply the steps described above to visualize, mask, produce an NDVI and
perform a supervised classification to the following image:
LC80440332016115LGN00

Change the appropriate arguments in the script to produce the following
outputs. Produce a screenshot of each one of them and submit them as
your report: ADD LINK

\begin{enumerate}
\def\labelenumi{\arabic{enumi}.}
\item
  An optimal RGB visualization of the input image using bands 5, 4, 3.
  Please modifiyi the stretch and gamma parameters to optimize
  visualization.
\item
  An NDBI image with a color palette from red, to yellow, to green. You
  can find the NDBI formula here:
  \url{https://www.linkedin.com/pulse/ndvi-ndbi-ndwi-calculation-using-landsat-7-8-tek-bahadur-kshetri}
\item
  A cloud mask prouduced based on an optimal threshold applied to the
  cloudscore image
\item
  An unsupervised classification with the number of clusters that best
  represent the land cover types in the original image.
\end{enumerate}

\begin{Shaded}
\begin{Highlighting}[]

\end{Highlighting}
\end{Shaded}

\end{document}
